\documentclass[a4paper,12pt]{article}

\usepackage[utf8]{inputenc}
\usepackage{amsmath}
\usepackage{graphicx}
\usepackage{hyperref}
\usepackage[margin=0.8in]{geometry}
\usepackage{amsthm}
\usepackage{amssymb}
\usepackage{float}
\usepackage{subfigure}
\usepackage{booktabs}
\usepackage{array}
\usepackage{algorithm}
\usepackage{algpseudocode}
\usepackage{listings}
\usepackage{xcolor}

% Code listing style
\lstset{
    basicstyle=\ttfamily\small,
    commentstyle=\color{gray},
    keywordstyle=\color{blue},
    stringstyle=\color{red},
    showstringspaces=false,
    breaklines=true,
    frame=single,
    captionpos=b
}

\title{\textbf{Network-on-Chip Design:\\Three-Dimensional Torus Topology with Adaptive Routing}}
\author{\textbf{Liye Wang\footnote{Student ID: 2023011411}, Youbo Shao\footnote{Student ID: 2023011412}}\\
        \textit{IIIS, Tsinghua University}}
\date{September 2025}

\begin{document}

\maketitle

\begin{abstract}
This report presents an implementation and performance evaluation of a three-dimensional torus Network-on-Chip (NoC) architecture with adaptive routing algorithms. We developed a 3D torus topology implementation within the Gem5/Garnet simulation framework, incorporating both deterministic dimension-order routing and adaptive routing mechanisms with escape virtual channels. Our implementation addresses key challenges in NoC design including deadlock avoidance, load balancing, and performance optimization in high-dimensional interconnection networks. Through experimentation across multiple traffic patterns and configuration parameters, we evaluate the performance characteristics of 3D torus architectures and analyze the trade-offs between different routing strategies and virtual channel allocation schemes.
\end{abstract}

\section{Introduction}

The ever-increasing demand for computational performance in multicore processors and system-on-chip (SoC) designs has driven the evolution of Network-on-Chip architectures from simple bus-based interconnects to sophisticated multidimensional topologies \cite{dally2001route, duato2002interconnection}. Three-dimensional torus networks represent a compelling solution for high-performance computing applications, offering superior bisection bandwidth, reduced diameter, and enhanced fault tolerance compared to traditional 2D mesh topologies \cite{scott2006blackwidow}.

However, the implementation of 3D torus NoCs presents challenges. The increased connectivity and bidirectional wraparound links introduce complex routing dependencies that can lead to deadlock conditions. Traditional dimension-order routing (DOR) algorithms, while simple and deadlock-free, often result in suboptimal load distribution and congestion hotspots \cite{glass1992turn}. Adaptive routing mechanisms, though promising for load balancing, require deadlock avoidance strategies such as escape virtual channels \cite{duato2002new}.

This work addresses these challenges through the implementation of a 3D torus NoC architecture with adaptive routing capabilities. Our primary contributions include:

\begin{enumerate}
    \item A complete 3D torus topology implementation in the Gem5/Garnet simulation framework with parameterizable dimensions and comprehensive port connectivity
    \item Implementation of adaptive routing algorithms incorporating escape virtual channel mechanisms for deadlock-free operation
    \item Performance evaluation across multiple traffic patterns, injection rates, and network configurations
    \item Analysis of virtual channel allocation strategies and their impact on network performance
    \item Comparative study of adaptive tie-breaking mechanisms and distance-based routing preferences
\end{enumerate}

The remainder of this report is organized as follows: Section 2 provides essential background on NoC architectures and routing algorithms. Section 3 details our 3D torus implementation and adaptive routing mechanisms. Section 4 presents our experimental methodology and configuration parameters. Sections 5-7 contain experimental results and analysis (to be completed upon data collection). Section 8 discusses the implications of our findings, and Section 9 concludes with future research directions.

\section{Background and Related Work}

\subsection{Network-on-Chip Architectures}

Network-on-Chip architectures have emerged as the dominant interconnection paradigm for multicore processors and complex SoCs. Unlike traditional bus-based systems, NoCs provide scalable, high-bandwidth communication through packet-switched networks with dedicated routing infrastructure \cite{benini2002networks}. The choice of network topology significantly impacts performance characteristics including latency, throughput, power consumption, and fault tolerance.

Mesh topologies, particularly 2D meshes, have been widely adopted in commercial multicore processors due to their simplicity and regular structure. However, as core counts increase, the limitations of 2D meshes become apparent: linear scaling of diameter with network size, limited bisection bandwidth, and uneven link utilization patterns.

\subsection{Three-Dimensional Torus Topologies}

Three-dimensional torus networks address many limitations of 2D mesh architectures through increased dimensional connectivity. A 3D torus with dimensions $X \times Y \times Z$ provides each router with six bidirectional ports (East/West, North/South, Up/Down), creating wraparound connections in all three dimensions. This topology offers several advantages:

\begin{itemize}
    \item \textbf{Reduced Diameter}: The network diameter scales as $O(\sqrt[3]{N})$ compared to $O(\sqrt{N})$ for 2D meshes
    \item \textbf{Enhanced Bisection Bandwidth}: Multiple dimensional cuts provide higher aggregate bandwidth
    \item \textbf{Load Distribution}: Wraparound links enable more uniform traffic distribution
    \item \textbf{Fault Tolerance}: Multiple paths between any source-destination pair improve resilience
\end{itemize}

Notable implementations include the Cray T3E supercomputer and IBM Blue Gene series, which demonstrated the practical benefits of 3D torus architectures in high-performance computing environments \cite{scott2006blackwidow}.

\subsection{Adaptive Routing Algorithms}

While deterministic routing algorithms guarantee deadlock freedom through restricted routing functions, they often lead to uneven network utilization and performance bottlenecks. Adaptive routing algorithms can dynamically select among multiple valid paths based on network conditions, potentially improving both throughput and latency.

Duato's escape channel theory provides a foundational framework for deadlock-free adaptive routing \cite{duato2002new}. The theory establishes that deadlock can be avoided by providing a set of escape channels that form an acyclic channel dependency graph. In practical implementations, this translates to reserving specific virtual channels for deterministic escape routing while allowing adaptive routing on remaining channels.

\subsection{Virtual Channel Flow Control}

Virtual channels represent a critical component of modern NoC designs, enabling improved throughput and quality-of-service guarantees \cite{dally1992virtual}. By multiplexing multiple logical channels over a single physical link, virtual channels can:

\begin{itemize}
    \item Reduce head-of-line blocking effects
    \item Enable adaptive routing with deadlock avoidance
    \item Provide traffic isolation for different priority classes
    \item Improve overall network utilization
\end{itemize}

The optimal allocation of virtual channels across different routing strategies remains an active research area, with trade-offs between hardware complexity, buffer requirements, and performance benefits.

\section{Implementation and Design}

\subsection{3D Torus Topology Implementation}

Our 3D torus implementation within the Gem5/Garnet framework provides a complete and parameterizable topology with the following key features:

\subsubsection{Topology Structure}
The topology supports arbitrary $X \times Y \times Z$ configurations through command-line parameters:
\begin{lstlisting}[language=bash, caption=3D Torus Configuration Parameters]
--topology=Torus3D
--torus-x=<X_dimension>
--torus-y=<Y_dimension>
--torus-z=<Z_dimension>
\end{lstlisting}

Each router in the 3D torus maintains six bidirectional ports with consistent naming conventions:
\begin{itemize}
    \item \textbf{East/West}: X-dimension connectivity
    \item \textbf{North/South}: Y-dimension connectivity
    \item \textbf{Up/Down}: Z-dimension connectivity
\end{itemize}

\subsubsection{Coordinate Mapping and Addressing}
Router coordinates are mapped to unique identifiers using a linear addressing scheme:
\begin{equation}
\text{RouterID} = z \cdot X \cdot Y + y \cdot X + x
\end{equation}

This mapping enables efficient coordinate extraction for routing decisions:
\begin{align}
z &= \lfloor \text{RouterID} / (X \cdot Y) \rfloor \\
y &= \lfloor (\text{RouterID} \bmod (X \cdot Y)) / X \rfloor \\
x &= \text{RouterID} \bmod X
\end{align}

\subsubsection{Link Connectivity}
The implementation creates comprehensive bidirectional links with proper wraparound connectivity. For each dimension, links are established between adjacent routers with modular arithmetic to handle torus wraparound:

\begin{algorithm}
\caption{3D Torus Link Creation}
\begin{algorithmic}[1]
\For{each dimension $d \in \{X, Y, Z\}$}
    \For{each router coordinate $(x,y,z)$}
        \State $\text{next\_coord} = (\text{coord} + 1) \bmod \text{dim\_size}$
        \State Create bidirectional link to next router
        \State Configure proper port directions
    \EndFor
\EndFor
\end{algorithmic}
\end{algorithm}

\subsection{Routing Algorithm Design}

\subsubsection{Deterministic Dimension-Order Routing}
Our baseline implementation employs deterministic dimension-order routing (DOR) with X-Y-Z priority ordering. This algorithm ensures deadlock freedom through strict dimensional ordering while considering torus wraparound characteristics.

For each dimension, the algorithm computes the optimal direction considering the torus topology:
\begin{align}
\text{forward\_distance} &= (\text{dest\_coord} - \text{curr\_coord} + \text{dim\_size}) \bmod \text{dim\_size} \\
\text{backward\_distance} &= (\text{curr\_coord} - \text{dest\_coord} + \text{dim\_size}) \bmod \text{dim\_size}
\end{align}

The algorithm selects the direction that minimizes hop count, with tie-breaking favoring forward direction for consistency.

\subsubsection{Adaptive Routing with Escape Virtual Channels}
Building upon the deterministic foundation, we implemented adaptive routing mechanisms that leverage Duato's escape channel theory. The adaptive routing algorithm operates as follows:

\begin{algorithm}
\caption{Adaptive Routing with Escape VCs}
\begin{algorithmic}[1]
\State \textbf{Input:} Source coordinates, destination coordinates, available VCs
\If{adaptive VCs available}
    \State Compute all valid adaptive directions
    \State Evaluate congestion and distance scores for each direction
    \State Select optimal direction based on combined metric
    \State Allocate adaptive virtual channel
\Else
    \State Fall back to deterministic escape routing
    \State Allocate escape virtual channel
\EndIf
\end{algorithmic}
\end{algorithm}

\subsubsection{Virtual Channel Allocation Strategy}
Virtual channels are partitioned into two categories:
\begin{itemize}
    \item \textbf{Escape VCs}: Reserved for deterministic DOR routing, ensuring deadlock freedom
    \item \textbf{Adaptive VCs}: Available for adaptive routing decisions based on network conditions
\end{itemize}

The allocation ratio between escape and adaptive VCs represents a critical design parameter, with implications for both performance and deadlock avoidance guarantees.

\subsubsection{Adaptive Tie-Breaking Mechanisms}
When multiple directions offer equivalent routing scores, our implementation provides configurable tie-breaking strategies:

\begin{itemize}
    \item \textbf{Dimension-First}: Prioritize X, then Y, then Z dimensions
    \item \textbf{Uniform Random}: Random selection among equivalent options
    \item \textbf{Z-First}: Prioritize Z dimension for improved 3D utilization
\end{itemize}

\subsubsection{Distance-Based Routing Preferences}
Our adaptive routing incorporates distance-based scoring with configurable coefficients:
\begin{equation}
\text{Score} = \alpha \cdot \text{Congestion} + \beta \cdot \text{Distance}
\end{equation}

where the distance coefficient $\beta$ can be tuned to favor either conservative (short remaining distance) or aggressive (long remaining distance) routing strategies.

\section{Experimental Methodology}

Our evaluation encompasses multiple experimental dimensions designed to characterize the performance characteristics of the 3D torus NoC architecture under various operating conditions.

\subsection{Simulation Framework}
All experiments are conducted using the Gem5 full-system simulator with the Garnet detailed network model. Gem5 provides cycle-accurate simulation of NoC behavior including router pipeline stages, virtual channel allocation, switch arbitration, and credit-based flow control.

\subsection{Network Configuration}
The baseline configuration employs a $4 \times 4 \times 4$ 3D torus with 64 nodes, representing a realistic scale for contemporary multicore processors. Key parameters include:

\begin{table}[H]
\centering
\caption{Baseline Network Configuration}
\begin{tabular}{@{}lr@{}}
\toprule
\textbf{Parameter} & \textbf{Value} \\
\midrule
Topology & $4 \times 4 \times 4$ 3D Torus \\
Number of Nodes & 64 \\
Virtual Networks & 3 \\
VCs per Virtual Network & 2-8 (variable) \\
Router Latency & 1 cycle \\
Link Latency & 1 cycle \\
Buffer Depth & 4 flits per VC \\
Flit Size & 16 bits \\
Simulation Cycles & 10,000 \\
\bottomrule
\end{tabular}
\end{table}

\subsection{Experimental Design}

Our experimental evaluation is structured around six primary investigation areas:

\subsubsection{Baseline Performance Characterization}
Establishes performance baselines using deterministic DOR routing across varying injection rates (0.01-1.0 packets/node/cycle) with transpose traffic pattern to stress the 3D torus characteristics.

\subsubsection{Synthetic Traffic Pattern Analysis}
Evaluates adaptive routing performance across eight distinct synthetic traffic patterns:
\begin{itemize}
    \item \textbf{Uniform Random}: Validates general-purpose performance
    \item \textbf{Transpose}: Tests worst-case dimensional ordering stress
    \item \textbf{Bit Complement}: Examines maximum-distance communication
    \item \textbf{Bit Reverse}: Evaluates pattern-specific optimizations
    \item \textbf{Tornado}: Stresses bisection bandwidth utilization
    \item \textbf{Shuffle}: Tests intermediate-distance routing efficiency
    \item \textbf{Neighbor}: Examines local communication patterns
    \item \textbf{Bit Rotation}: Validates mixed-distance performance
\end{itemize}

\subsubsection{Adaptive Tie-Breaking Strategy Comparison}
Systematic evaluation of three tie-breaking mechanisms (x\_first, uniform\_random, z\_first) to identify optimal strategies for 3D torus architectures.

\subsubsection{Virtual Network Injection Analysis}
Investigation of traffic injection strategies across different virtual networks to understand performance implications of traffic classification and isolation.

\subsubsection{Distance Coefficient Optimization}
Parametric study of distance-based routing preferences with coefficients ranging from -4 (strong conservative bias) to +4 (strong aggressive bias) to identify optimal load balancing strategies.

\subsubsection{Virtual Channel Configuration Optimization}
Comprehensive analysis of VC allocation strategies examining 11 distinct configurations:
$(VCs\_per\_VNet, Escape\_VCs) \in \{(1,1), (2,1), (2,2), (4,1), (4,2), (4,3), (4,4), (8,1), (8,2), (8,4), (8,8)\}$

This analysis investigates the fundamental trade-off between adaptive routing flexibility and deadlock avoidance guarantee strength.

\subsection{Performance Metrics}
Our evaluation focuses on four primary performance metrics:

\begin{itemize}
    \item \textbf{Throughput}: Maximum sustainable injection rate before saturation
    \item \textbf{Average Packet Latency}: End-to-end packet delivery time
    \item \textbf{Network Latency}: Time spent traversing network infrastructure
    \item \textbf{Average Hop Count}: Path length efficiency metric
\end{itemize}

\section{Experimental Results and Analysis}

\textit{[This section will be completed upon data collection and analysis]}

\subsection{Baseline Performance Characterization}

\textit{[Results of deterministic routing performance across injection rates]}

\subsection{Synthetic Traffic Pattern Performance}

\textit{[Comparative analysis of adaptive vs. deterministic routing across eight traffic patterns]}

\subsection{Adaptive Tie-Breaking Strategy Analysis}

\textit{[Performance comparison of different tie-breaking mechanisms]}

\subsection{Virtual Network Injection Strategy Evaluation}

\textit{[Analysis of traffic injection patterns and their performance implications]}

\subsection{Distance Coefficient Optimization Results}

\textit{[Parametric study results for distance-based routing preferences]}

\subsection{Virtual Channel Configuration Analysis}

\textit{[Comprehensive evaluation of VC allocation strategies and their performance trade-offs]}

\section{Discussion}

\textit{[This section will be completed upon analysis completion]}

\subsection{Performance Trade-offs and Design Implications}

\textit{[Discussion of key trade-offs identified through experimentation]}

\subsection{Scalability Considerations}

\textit{[Analysis of scalability implications for larger 3D torus configurations]}

\subsection{Practical Implementation Considerations}

\textit{[Discussion of real-world implementation challenges and solutions]}

\section{Conclusion and Future Work}

This report presents an implementation and evaluation of 3D torus NoC architectures with adaptive routing mechanisms. Our implementation demonstrates the viability and performance characteristics of routing algorithms in high-dimensional network topologies.

\textit{[Specific conclusions will be added upon completion of experimental analysis]}

Key areas for future investigation include:
\begin{itemize}
    \item Extension to fault-tolerant routing mechanisms
    \item Integration with cache coherence protocols
    \item Power consumption analysis and optimization
    \item Scalability evaluation for larger dimensional configurations
    \item Hardware implementation and synthesis studies
\end{itemize}

The developed framework provides a robust foundation for continued research in advanced NoC architectures and routing algorithms.

\section*{Acknowledgments}

The authors thank the course instructor and TAs for their helpful guidance.

\nocite{*}
\bibliographystyle{IEEEtran}
\bibliography{reference}


\end{document}
